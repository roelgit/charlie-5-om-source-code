\documentclass[11pt]{beamer}
%Gummi|065|=)
\title{\textbf{Force directed graph drawing}}
\author{Roxanne Giling\\
		Laurens Post\\
		Jeroen Hoegen Dijkhof}
\date{}
\begin{document}

\begin{frame}
    \frametitle{\maketitle}
\end{frame}

\begin{frame}
    \frametitle{Wat is force directed graph drawing}
    \begin{itemize}
        \item{Algoritme voor het tekenen van grafen}
        \item{Aantrekkende en afstotende kracht}
        \item{Simuleert beweging}
        \item{Multidimensionaal}
    \end{itemize}
\end{frame}

\begin{frame}
    \frametitle{Algoritme dat we testen}
    \begin{itemize}
        \item{Gepubliceerd door Eades in 1984}
        \item{Gemaakt voor het tekenen van relatief kleine grafen}
        \item{Krachten optellen}
        \begin{itemize}
            \item{
                 $d_{u,v}$ = afstand tussen u en v
            }
            \item{
                \begin{math}
                    aantrekking_{v,u} = c_1 * log(\frac{d_{u,v}}{c_2})
                \end{math}
            }
            \item{
                \begin{math}
                    afstoting_{v,u} = \frac{c_3}{d_{v,u}^2}
                \end{math}
            }
            \item {
                \begin{math}
                    beweging_u = c_4 * \sum_{i=1}^V{(aantrekking_{u,V_i} + afstoting_{u,V_i})}
                \end{math}
            }
        \end{itemize}
    \end{itemize}
\end{frame}

\begin{frame}
    \frametitle{Vooraf verzamelde data}
    //TODO
\end{frame}

\begin{frame}
    \frametitle{Hypotheses}
    //TODO
\end{frame}

\begin{frame}
    \frametitle{Vragen}
    U kunt nu vragen stellen als u die heeft.
\end{frame}

\end{document}
