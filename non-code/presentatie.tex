\documentclass[16pt]{beamer}
%Gummi|065|=)
\title{\textbf{Force directed graph drawing}}
\usetheme{Berlin}
\usecolortheme{beaver}
\author{Roxanne Giling\\
		Laurens Post\\
		Jeroen Hoegen Dijkhof}
\date{}

\expandafter\def\expandafter\insertshorttitle\expandafter{%
  \insertshorttitle\hfill%
  \textbf{\insertframenumber}\,/\,\inserttotalframenumber}

\begin{document}

\begin{frame}
    \titlepage
\end{frame}

\begin{frame}
    \frametitle{Introductie: wat is force directed graph drawing}
    \begin{itemize}
        \item{Algoritme voor het tekenen van grafen}
        \item{Aantrekkende en afstotende kracht}
        \item{Simuleert beweging}
        \item{Multidimensionaal}
    \end{itemize}
\end{frame}

\begin{frame}
    \frametitle{Algoritme dat we testen}
    \begin{itemize}
        \item{Gepubliceerd door Eades in 1984}
        \item{Gemaakt voor het tekenen van relatief kleine grafen}
        \item{Krachten optellen}
        \begin{itemize}
            \item{
                 $d_{u,v}$ = afstand tussen u en v
            }
            \item{
                \begin{math}
                    aantrekking_{v,u} = c_1 * log(\frac{d_{u,v}}{c_2})
                \end{math}
            }
            \item{
                \begin{math}
                    afstoting_{v,u} = \frac{c_3}{d_{v,u}^2}
                \end{math}
            }
            \item {
                \begin{math}
                    beweging_u = c_4 * \sum_{i=1}^V{(aantrekking_{u,V_i} + afstoting_{u,V_i})}
                \end{math}
            }
        \end{itemize}
    \end{itemize}
\end{frame}

\begin{frame}
    \frametitle{Hypotheses}
    \begin{itemize}
        \item{Looptijd tegenover aantal iteraties is lineair}
        \item{Som van de lengte van de ribben neemt lineair toe bij het vergroten van $c_3$}
        \item{Som van de lengte van de ribben neemt lineair af bij het verlagen van $c_1$}
        \item{Looptijd van algoritme neemt kwadratisch toe naarmate het aantal vertices stijgt}
    \end{itemize}
\end{frame}

\begin{frame}
    \frametitle{Criteria}
    \begin{itemize}
        \item{Looptijd}
        \item{Looptijd per iteratie}
        \item{Totale lengte van de ribben}
    \end{itemize}
\end{frame}

\begin{frame}
    \frametitle{Testdata}
    \begin{itemize}
        \item{Internet routing table}
        \begin{itemize}
            \item{Internet is een graaf!}
        \end{itemize}
        \item{Gegenereerde data}
        \begin{itemize}
            \item{Per hypothese te genereren}
            \item{Bijvoorbeeld veel edges per vertex}
        \end{itemize}
    \end{itemize}
\end{frame}

\begin{frame}
    \frametitle{Zaken die we niet behandelen}
    \begin{itemize}
        \item{Andere algoritmes}
        \begin{itemize}
            \item{Barycentrisch}
            \item{Geoptimaliseerde algoritmes voor grote grafen}
        \end{itemize}
        \item{Bipartiet / niet-partitiete grafen}
        \item{Spoorkaart}
        \begin{itemize}
            \item{Wilden we doen, maar data is lastig te verkrijgen}
        \end{itemize}
    \end{itemize}
\end{frame}


\begin{frame}
    \frametitle{Vragen}
    Bedankt voor uw aandacht. U kunt nu vragen stellen als u die heeft.
\end{frame}

\end{document}
